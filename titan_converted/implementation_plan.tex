\documentclass{article}
\usepackage[utf8]{inputenc}
\usepackage{graphicx}
\usepackage{listings}
\usepackage{hyperref}
\usepackage{amsmath}
\usepackage{geometry}

\title{LLaMA-to-Titans Architecture Conversion Plan}
\author{Devin AI}
\date{\today}

\begin{document}

\maketitle

\section{Executive Summary}
This document outlines the technical approach for converting the LLaMA 7B 3.3 transformer model to implement the Titans architecture. The conversion focuses on implementing Titans' three-component memory system while optimizing for specific hardware constraints (3x NVIDIA RTX 3090 GPUs, 64GB total VRAM).

\section{Current Architecture (LLaMA 7B 3.3)}
\subsection{Core Components}
\begin{itemize}
    \item Model Parameters:
    \begin{itemize}
        \item 32 transformer layers
        \item 32 attention heads
        \item Dimension size: 4096
        \item Uses RMSNorm for layer normalization
        \item Maximum sequence length: 4096 tokens
    \end{itemize}
    \item Key Modules:
    \begin{itemize}
        \item Attention mechanism with rotary positional embeddings (RoPE)
        \item Parallel implementation via fairscale
        \item Key-value caching for efficient inference
    \end{itemize}
    \item Memory Management:
    \begin{itemize}
        \item Current parameters: max\_batch\_size=32, max\_seq\_len=2048
        \item Uses model parallel processing for memory distribution
        \item Implements cache\_k and cache\_v for attention
    \end{itemize}
\end{itemize}

\section{Target Architecture (Titans)}
\subsection{Three-Component Memory System}
\begin{itemize}
    \item Core Module:
    \begin{itemize}
        \item Modified attention mechanism from traditional transformers
        \item Maintains accurate dependency modeling
        \item Handles immediate context processing
    \end{itemize}
    \item Long-term Memory:
    \begin{itemize}
        \item Neural memory module for historical context
        \item Specialized for maintaining long-range dependencies
        \item Implements efficient retrieval mechanism
    \end{itemize}
    \item Persistent Memory:
    \begin{itemize}
        \item Task-specific knowledge storage
        \item Optimized for specialized information retention
        \item Supports 2M+ context window size
    \end{itemize}
\end{itemize}

\section{Implementation Strategy}
\subsection{Memory Distribution Across GPUs}
Total VRAM: 64GB across 3x RTX 3090 GPUs
\begin{itemize}
    \item GPU 1 (Core Module): $\sim$22GB
    \begin{itemize}
        \item Primary attention mechanisms
        \item Token embeddings
        \item Layer normalization
    \end{itemize}
    \item GPU 2 (Long-term Memory): $\sim$21GB
    \begin{itemize}
        \item Historical context storage
        \item Retrieval mechanisms
        \item Cache management
    \end{itemize}
    \item GPU 3 (Persistent Memory): $\sim$21GB
    \begin{itemize}
        \item Task-specific knowledge base
        \item Specialized storage systems
        \item Integration logic
    \end{itemize}
\end{itemize}

\subsection{Architectural Modifications}
\begin{itemize}
    \item Core Module Adaptations:
    \begin{itemize}
        \item Modify attention mechanism to support larger context windows
        \item Implement efficient memory access patterns
        \item Optimize for parallel processing
    \end{itemize}
    \item Long-term Memory Implementation:
    \begin{itemize}
        \item Design neural memory module
        \item Implement retrieval mechanisms
        \item Optimize for historical context maintenance
    \end{itemize}
    \item Persistent Memory Integration:
    \begin{itemize}
        \item Create specialized storage system
        \item Implement knowledge integration logic
        \item Optimize for task-specific retention
    \end{itemize}
\end{itemize}

\section{Testing and Validation}
\subsection{Unit Tests}
\begin{itemize}
    \item Core Module:
    \begin{itemize}
        \item Attention mechanism functionality
        \item Memory access patterns
        \item Performance benchmarks
    \end{itemize}
    \item Long-term Memory:
    \begin{itemize}
        \item Historical context retention
        \item Retrieval accuracy
        \item Memory management efficiency
    \end{itemize}
    \item Persistent Memory:
    \begin{itemize}
        \item Knowledge storage integrity
        \item Integration effectiveness
        \item Task-specific performance
    \end{itemize}
\end{itemize}

\subsection{Integration Tests}
\begin{itemize}
    \item End-to-end system validation
    \item Performance metrics:
    \begin{itemize}
        \item Context window size verification (2M+ tokens)
        \item Memory usage monitoring
        \item Processing speed benchmarks
    \end{itemize}
    \item Cross-component interaction testing
\end{itemize}

\section{Performance Optimization}
\subsection{Memory Management}
\begin{itemize}
    \item VRAM optimization strategies:
    \begin{itemize}
        \item Gradient checkpointing
        \item Memory-efficient attention patterns
        \item Optimized cache management
    \end{itemize}
    \item Load balancing across GPUs
    \item Memory access pattern optimization
\end{itemize}

\subsection{Computational Efficiency}
\begin{itemize}
    \item Parallel processing optimization
    \item Cache utilization improvements
    \item Batch size optimization
\end{itemize}

\section{Implementation Timeline}
\begin{itemize}
    \item Phase 1: Core Module Adaptation (1-2 weeks)
    \item Phase 2: Long-term Memory Implementation (1-2 weeks)
    \item Phase 3: Persistent Memory Integration (1-2 weeks)
    \item Phase 4: Testing and Optimization (1-2 weeks)
\end{itemize}

\section{Conclusion}
This implementation plan outlines the technical approach for converting LLaMA 7B 3.3 to the Titans architecture. The plan ensures efficient utilization of available hardware while maintaining model performance and implementing all key features of the Titans architecture.

\end{document}
